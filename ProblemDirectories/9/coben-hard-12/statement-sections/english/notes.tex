We have 3 players with current scores $p = [60, 40, 20]$ and bonus options $s = [100, 50, 25]$. There are $3! = 6$ ways to assign the 3 bonuses, and in each assignment, the player with the lowest total $(p_i + s_{\pi(i)})$ is eliminated. Enumerating:   

1. $[160,90,45] \to $ Player 3 is eliminated.  

 
2. $[160,65,70]  \to $ Player 2 is eliminated. 

3. $[110,140,45]  \to $ Player 3 is eliminated.  

4. $[110,65,120]  \to $ Player 2 is eliminated. 

5. $[85,140,45]  \to $ Player 3 is eliminated. 

6. $[85,90,120]  \to $ Player 1 is eliminated. 


\textbf{Player 1} is eliminated in 1 of 6 permutations $(\frac{1}{6})$.   

 
\textbf{Player 2} is eliminated in 2 of 6 permutations $(\frac{1}{3})$.    


\textbf{Player 3} is eliminated in 3 of 6 permutations $(\frac{1}{2})$.

A tie is possible for the last test case; if we allocate the score bonus $5$ to player 1 and the score bonus $4$ to player 2, both end with a total score of $8$. Therefore, we begin the procedure and uniformly pick one of them at random, which means they each have a $\frac{1}{2}$ chance of being eliminated if those score bonuses are allocated in that manner. The probabilities of elimination over all permutations are $\frac{3}{4}$ for player 1 and $\frac{1}{4}$ for player 2.